Given a $n\times n$ multiplication table, how many unique numbers $M(n)$
are there?
\begin{equation*}
    M(n) = \left|\{ab: 1 \leq a,b \leq n\}\right|
\end{equation*}

\noindent Let $\delta(n)$ count the entries of the column $n$ that
appear in the $(n-1)\times(n-1)$ table.
\begin{equation*}
    M(n) = M(n-1) + \(n - \delta(n)\)
\end{equation*}

\noindent Consider the factorization $ab = (ij)(gh)$. If it is possible to form
some $mn = (ih)(gj)$ such that $ih, gj < n$ then $mn$ has already appeared in a
smaller tablea. Now, $ih < n$ iff $i < g$ and $gj < n$ iff $j < n/g = h$. So
$\delta(n)$ can be computed by counting unique values $i,j$ for each divisor
$g \leq \sqrt{n}$ of $n$.

\inputminted[linenos, frame=lines]{python}{Algorithms/Erdos\ Multiplication\ Table/erdos_multiplication_table.py}